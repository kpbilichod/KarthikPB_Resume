\documentclass{resume} % Use the custom resume.cls style
\usepackage{graphicx}
\usepackage{hyperref}

\usepackage[left=0.75in,top=0.6in,right=0.75in,bottom=0.6in]{geometry} % Document margins
\newcommand{\tab}[1]{\hspace{.2667\textwidth}\rlap{#1}}
\newcommand{\itab}[1]{\hspace{0em}\rlap{#1}}

\name{Karthik P Bilichod} % Your name

\begin{document}


\begin{rSection}{ }

H.No-605, 17th main, 10th cross \hfill  Contact No. : 8971278593 \\
Padmanabhanagar  \hfill Email ID : karthikpk23@gmail.com \\
Bengaluru - 560070\\
Karnataka\\
\end{rSection}

\begin{center}
\hspace{10 cm}
\includegraphics{karthik}
\end{center}

\vskip 0.5in

\begin{rSection}{Objective}
 I am interested to work in the field of Embedded systems, Robotics(especially aerial robotics), RF and Antennas. I will definitely work hard and give my best to solve problems which will help the society.
\end{rSection}

\vskip 0.5in

\begin{rSection}{Education}
\centering
\begin{tabular}{|l|l|l|l|l|}
\hline
{\bf Course} & {\bf Discipline} & {\bf Institution} & {\bf Year of passing} & {\bf Score} \\
&&&&\\
\hline
 School & Class-10th CBSE & Kendriya Vidyalaya Hospet & 2014 & 9.6 CGPA\\
 &&&&\\
 \hline
 Pre University & Class-12th CBSE & Sharada Vidyaniketana & 2016 & 92.25(PCMB)\\
  & & Public School & &Percent\\
  &&&&\\
  \hline
  B.Tech & Electronics and & PES University & 2020 & 8.3 CGPA\\
  &Communication&&&Sem 1 to Sem 5\\
  & Engineering&&&\\
  \hline
\end{tabular}
\end{rSection}

\pagebreak

\begin{rSection}{Projects}
\begin{enumerate}
    \item {\bf Ball catching Robot} during Summer internship at Microsoft Innovation Lab PES University under the guidance of Dr.Venkatarangan MJ and Prof Rajasekar M. Used Firebird-V platform for catching and OpenCV for trajectory prediction.
    \item {\bf Self balancing Robot} using PID control as a mini project for the course Introduction to Robotics under the guidance of Dr.Venkatrangan MJ.
    \item Faster computation of {\bf Inverse of huge matrices} using recurrent neural networks as a mini project for course Machine Learning under the guidance of Dr.Koshy K George.{\bf (In progress)}
    \item {\bf Determination of array factor for Superdirective characterstics} as a research project under the guidance of Dr.Saumya Adhikari SA.
    \item A {\bf Drone} for Defence applications as a hobby project.{\bf (In progress)}
    \item Establishing fast communication between multiple ESP32 modules using {\bf WiFi Direct} (without internet)  under the guidance of Dr.Manikandan J at CORI, PES University.
    \item {\bf Sun tracking solar panel} using AT89C51 microcontroller and LDRs as a mini project for the course Microcontrollers.
    \item {\bf Cell Phone Detector} using LM358 Dual OpAmp as a mini project for the course Linear Integrated Circuits.
    \item {\bf Fruit Plucking Robot} for JED-I(Joy of Engineering and design) lab using Arduino Uno and image processing using OpenCV.
\end{enumerate}
\end{rSection}

\vskip 0.5in

\begin{rSection}{Training and Internships}
\begin{itemize}
    \item {\bf PNM Satellite Design course} held at PES University by Prof. Sharan Asundi from Tuskegee University.
    \item {\bf Compu music} course to produce and compose music using software Chuck.
    \item {\bf IoT workshop} held at PES University by i3indya technologies.
    \item Interned at {\bf CORI}( Crucible for Research and Innovation), PES University.
    \item Interned at {\bf Microsoft Innovation Lab}, PES University.\\

\end{itemize}
\end{rSection}

\vskip 0.5in

\begin{rSection}{Research Publications}
\begin{enumerate}
    \item {\bf To synthesize an Array Factor for Superdirective chracterstics of a broadside Antenna}. Have successfully completed and finalized the array factor and the work is about to be published. It is being done under the guidance of {\bf Dr.Saumya Adhikari SA}.\\
\end{enumerate}
\end{rSection}

\end{document}

